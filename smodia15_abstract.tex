\documentclass[12pt]{amsart}
\usepackage{eulervm}
\renewcommand{\rmdefault}{ppl}
\newcommand{\gimli}{\texttt{gimli}}
\newcommand{\ie}{\textit{i.e.}}
\newcommand{\eg}{\textit{e.g.}}

\begin{document}

\title{Multiscale segmentation of methylation data}
\author{Emanuele Raineri}
\date{\today}
\maketitle

DNA Methylation has a strong local correlation,
in the sense that the level at one position predicts 
very well the observations nearby; this suggests that it should be possible
to build compressed representations of the data in which contiguous loci 
are collected into segments.

In this poster, I outline one efficient way of computing 
segmentations of methylation values determined by 
whole genome bisulfite sequencing (WGBS).
A segmentation joins adjacent positions
which have similar properties, while the boundary between segments 
indicates more or less abrupt changes
in the signal.

Changes in methylation can be observed both at the level of a single promoter
region  and when looking at stretches
of millions of bases.
This brings about contrasting requirements when analyzing this kind of data: 
large scale (slow) variations are 
significant and must be looked at, but a rough
running average with a window of many megabases might destroy interesting
local detail. Here I present  a software (\gimli{}) which finds  
segmentations at multiple scales.
%One salient aspects which must be considered when it comes to 
%analyzing methylation data is that this epigenetic mark seems 
%to have effect at various genomic scales 
%(ranging from hundred of bases to megabases), hence 
%one nees a method that works at many scales. 
%The probabilistic approach I introduce here 
%can be used to calculate automatic segmentations at different scales.
I also show how this 
kind of statistical modeling is expedient for 
visualization, comparison of samples,
and demarcation of unusual tracts in 
order to direct further analysis 
(\eg{} regions corresponding to stark changes).

The idea underpinning this method is, in its essence, to optimize  
a function which depends on the coordinates
of each block as well as on the spatial distribution of the CpGs.
The search for optima is carried out
in a greedy fashion, hence there is no guarantee
that it finds a global maximum: nevertheless, the approximated results are good enough
to be useful.
\end{document}
