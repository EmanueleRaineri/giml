\documentclass[12pt]{amsart}
\usepackage{graphicx}
\newcommand{\lik}{\ensuremath{\mathcal{L}}}
\newcommand{\gimli}{\texttt{gimli}}

\begin{document}
\title{Multiscale segmentation of methylation data}
\author{Emanuele Raineri}
\date{\today}
\maketitle

\begin{abstract}
%Since it is possible to decorate the human genome with tracks of numerical
%values produced by methods for measuring all sorts of physico-chemical interactions 
%(chromatin modification,
%Hi-C, whole genome bisulfite sequencing, and more), 
%it is useful to implement software 
%which can quickly summarize the data and extract patterns to help 
%automating at least  part of the biological analysis. 
In this paper, I show one efficient way of computing 
segmentations of methylation values determined by 
whole genome bisulfite sequencing
A segmentation joins adjacent position
which have similar properties, while the boundary between segments 
indicates more or less abrupt changes
in the signal which might relate to biological mechanisms. 
One salient aspects which must be considered when it comes to 
analyzing methylation dataset is that this epigenetic mark seems 
to have effect at various genomic scales 
(ranging from hundred of bases to megabases), hence a multi scale method 
is called for. 
The probabilistic method I introduce here 
coupled with a known procedure previously
used for copy number detection
can be used to calculate automatic segmentations at different scales.
I also show how this 
kind of statistical modeling is expedient for 
visualization, comparison across samples,
and demarcation of unusual regions in 
order to direct further analysis (i.e. regions with stark variations in the signal).
\end{abstract}

\section{Introduction}

Whole genome Bisulfite sequencing (WGBS) allows measurements of methylation 
status with single base resolution across an entire genome. 
Methylation has a strong local correlation,
in the sense that the level at one position predicts 
very well the observations nearby; this suggests that it should be possible
to build compressed representations of the data in which contiguous loci 
are collected into segments.
This local correlation is depicted in figure~\ref{fig_corr},
and it has been observed indipendently many times ( for example
see \cite{bsmooth} and references therein).

\begin{center}
\begin{figure}\label{fig_corr}
\includegraphics[width=7cm,angle=270]{out.correla.eps}
\caption{This plot is produced by extracting $1$ million random pairs of CpG loci;
storing the pairs in $20$ different bins  according to the distance between the members
of the pair (from $0-100$ to $1900-2000$); and computing the correlation in each bin,
using the pairs contained in it.}
\end{figure}
\end{center}

 
Changes in methylation can be observed both at the level of a single promoter
region (hundreds of bases, \cite{methylseekr}) and when looking at stretches
of millions of bases (\cite{largeblocks} ).
This brings about contrasting requirements when analyzing this kind of data: large scale (slow) changes are 
significant and must be looked at, but a rough
running average with a window of many megabases might destroy interesting
local detail. Here I present  a software (\gimli) which computes a multiscale segmentation 
of DNA methylation 
data acquired through WGBS and helps making sense of the data and compare 
it across samples.

A segmentation is a statistical model of a dataset  which takes into account its spatial distribution:
in particular, adjacent points with similar properties are grouped together.
In general, when building a statistical model one tries to find a reasonable 
trade off between goodness of fit and complexity.
In our case these two aspects have a very simple interpretation : they both typically 
increase with the number of segments $N_S$ used to describe the 
methylation data.  
At one extreme, representing the complete dataset with one big segment would 
give a very simple model with, but likely a very bad fit;  
at the other using many segments of length $1$ would normally produce
a well fitting but very complex summary of the data.
The complexity of the model is linked to the typical scale of the segments used
in the sense that if the typical block is very long one needs few of them to 
cover the genomic region under consideration.

\section{The algorithm}
\subsection{Blocks and their likelihood}
The  input to \gimli{} is a set of $N$ methylated positions 
which do not specifically need to be CpGs.
In the rest of the paper, though, I will refer to CpGs and 
call {\em block} or , equivalently, {\em segment}
both a single CpG or a set of contiguous CpGs. 
Note that when I say contiguous I mean that they are one
after the other along the genome, not necessarily that they are
close in genomic coordinates. The coordinates of the CpGs are 
indexed by $j=1,\dots,N$ : for example, for human chromosome $1$ 
one has $N = XXX $ in the human reference genome version $37$.
This is the maximum number that a WGBS experiment can measure. In practice,
due to the vagaries of the sequencing protocol, even at a decent read depth
some of them are not covered. 
The number of nonconverted (respectively, converted) reads mapping at 
position $j$ is $\overline{n}_j$ (respectively, $n_j$). 


The output of \gimli{} is a set of blocks 
covering the initial positions 
such that each block contains CpGs which are similarly
methylated. The strength with which
one imposes the similarity constraint
is controlled by a coarseness parameter $\lambda$.

Internally the program  mantains a list of the current  
blocks (indexed by $i=1,\dots,N_S$), 
which at the beginning 
coincides with the loci given in input. 
( i.e.  $N_S=N$ and each block has length $1$). 
To each segment $i$ there is associated a $\hat{\theta}_i$ which
is the maximum likelihood estimation of the methylation given the counts
at each position in the segment. 
For example, if segment $i$ includes the CpGs from $j_1$ to $j_2$
$\hat{\theta}_i$ is:

\[
\hat{\theta}=\frac{\overline{n}}{n + \overline{n}}
\]

where

\[
\overline{n}=\sum_{j=j_1}^{j_2} \overline{n}_j
\]

and

\[
n=\sum_{j=j_1}^{j_2} n_j
\]

The likelihood of such a block is defined as follows:

\begin{equation}
\lik_i=\sum_{j=j_1}^{j_2} 
\log {\overline{n}_j+n_j \choose \overline{n}_j} +
	{\overline{n}_j}\log\hat{\theta}_j+
	n_j\log(1-\hat{\theta}_j)
\end{equation}
\label{loglik}

hence it is the sum of the likelihoods of each locus in the block. I will use a boldface
$\pmb{\theta}_i$ to refer to the vector of the methylation values of all the loci belonging to 
block $i$.
The total likelihood of the segmentation is in turn given by 
$\mathcal{L}=\sum_i\mathcal{L}_i$.

\subsection{merging blocks}

The way the program works is by 
scanning repeatedly the list of blocks and evaluate whether any pair 
of adjacent ones can be conveniently merged into one.  
This is decided by checking whether the loss in 
likelihood consequent to the merging is compensated 
(or more than compensated)
by the reduction in complexity achieved by it.

The loss in 
total likelihood that we get if we merge a pair $(i,i+1)$ of 
adjacent segments into one is:

\[\mathcal{L}_{i,i+1}-\mathcal{L}_i-\mathcal{L}_{i+1}\]

where $\mathcal{L}_{i,i+1}$ is the likelihood of a segment which includes
all the loci in segments $i$ and $i+1$.
Notice that it can be  $\mathcal{L}_{i,i+1} \leq \mathcal{L}_i+\mathcal{L}_{i+1}$ and hence merging  
can introduce a loss in total 
likelihood. This loss is counterbalanced, though, by the fact the the number 
of segments decreases by $1$. To take into account this decrease in 
complexity we consider an adjusted likelihood change:

\[\Delta  \lik_{i,i+1} = -\mathcal{L}_i-\mathcal{L}_{i+1}+\mathcal{L}_{i,i+1} + \lambda\]

Once it has computed 
$\Delta  \lik_{i,i+1}$ for all $i$, the algorithm selects the maximum 
one : if this is positive, it conjoins the corresponding segments into one.

The software repeats the scanning and merging until it is not possible to 
merge any more pair; after that it can either increment $\lambda$ and try again, 
or give up, print the segmentation and exit, depending on the command line parameters.

The above procedure is similar to the one
described in the context of copy number variation detection in \cite{vega} 
( which, in turn, us an adaptation 
of a 2 dimensional image analysis algorithm due to Mumford and Shah, 
\cite{mumfordshah}) except that the probabilistic method  described
here is more suitable for analyzing methylation data. This is because it takes into account 
that the
uncertainty with which we know $\hat{\theta}_i$ is higher when the read depth is low and vice versa.
This has an effect on the decision of merging adjacent blocks. For example in \ref{fig_delta_cov} I plot
$\Delta  \lik$  (vertical axis)for two segments at $\theta=0.8$ and $\theta=0.45$ measured at different read depth
(horizontal axis).

\begin{figure}\label{fig_delta_cov}
\includegraphics[width=7cm,angle=270]{fig_delta_cov.eps}
\caption{$\Delta  \lik$ at fixed methylation difference depends on read depth.}
\end{figure}

Methylated sites are irregularly spaced along the genome and the variability 
introduced by the sequencing process usually increases the distance 
between adjacent sites.
To avoid building segments which span long regions where no CpGs exist, one 
can multiply the difference in likelihoods by a sigmoidal penalty term of the
form \[\rho=\frac{D1}{1+\exp(-D_{i,i+1}/D0)}-\frac{D_1}{2}+1\] (see \ref{figrho}) 
where $D_{i,i+1}$ is the distance 
between the the right end of segment $i$ and the left end of segment $i+1$ and 
$D_0,D_1$ are constants which  I set heuristically to  $D_0=10000,D_1=100$. 

\begin{figure}\label{figrho}
\includegraphics[width=7cm,angle=270]{figrho.eps}
\caption{Empirical penalty factor $\rho$}
\end{figure}

Hence the final expression for the adjusted likelihood change is :

\begin{equation}
\Delta \lik = \rho ( -\mathcal{L}_i-\mathcal{L}_{i+1}+\mathcal{L}_{i,i+1} )  +\lambda
\end{equation}

\section{Results}

First of all, reassuringly, the segments created by \gimli{} to have lower methylation
variability than random segments of the same length extracted from the input dataset. 
This is shown in \ref{fig_variance}.

\begin{figure}\label{fig_variance}
\includegraphics[width=7cm,angle=270]{fig_variance.eps}
\caption{Variance across segments of $15$ contiguous CpGs when chosen at random or 
computed with \gimli{}. Methylation values from }
\end{figure}

I can now refine quantitatively the assertion that the complexity of the model
is linked to the typical lengths of the segments : as figure \ref{fig_boxplot_size} 
shows the median length of the blocks increases with $\lambda$. This is 
accompanied by a decrease in likelihood of the segments (\ref{fig_boxplot_lik})

\begin{figure}\label{fig_boxplot_size}
\includegraphics[width=7cm,angle=270]{fig_boxplot_size.eps}
\caption{Median length of blocks is higher at higher $\lambda$s.}
\end{figure}

\begin{figure}\label{fig_boxplot_lik}
\includegraphics[width=7cm,angle=270]{fig_boxplot_lik.eps}
\caption{Likelihood of blocks diminishes with $\lambda$.}
\end{figure}

While one can verify that the numbers computed by the algorithm are in 
accordance with the description
given in this paper, the purpose of \gimli{} is to help 
detect automatically interesting biological patterns in methylation data, 
and this can't be tested in any 
precise sense. To show that \gimli{} is a useful tool
though, I will give $4$ examples of analyses which can be produced very quickly
using it. 

\subsection{PMD}



\subsection{Jumps}

First I'll show how the boundary of the segments, also known as the breakpoints
can correspond to landmarks in the genome. To this purpose, I considered the 
$4$ Blueprint monocytes \verb=C000S5A1bs=,\verb=C0010KA2bs=,\verb=C001UYA3bs= and \verb=C004SQ51= 
I computed their segmentation for $\lambda=10$ and looked for all the triples of consecutive segments
with corresponding vectors methylation values $\pmb{\theta}_{1},\pmb{\theta}_{2},\pmb{\theta}_{3}$
where 
\begin{itemize}
\item{} $\min \theta_{1i} \geq \max \theta_{2i} + 0.25$
\item{} $\min \theta_{3i} \geq \max \theta_{2i} + 0.25$
\item{} all the segments contain at least $5$ CpGs.
\item{} they must appear in all the 4 samples
\end{itemize}

These conditions correspond to regions where the methylation goes from high to low to high again,
in jumps of at least $0.25$.
See for example \ref{fig6} 

In this way I obtain $143$ such triplets. Looking at their central segmentes
(those in a comparatively low methylation states) I find that their median length is
$252$bp (minimum length $39$, maximum $252$). When I intersect this central segments with the Blueprint
annotation, I find than $94$ overlap a known transcript, suggesting that regions with stark 
changes in methylation
might sistematically correspond to regulated parts of the genome.

\begin{figure}\label{fig6}
\includegraphics[width=7cm,angle=270]{fig6.eps}
\caption{An example of high-low-high pattern found in chromosome $1$ by \gimli{}}
\end{figure}

\subsection{Differentially methylated regions}

\gimli{} can be used to catch differentially methylated regions across pair of samples
or pair of groups of samples.
As an example, in what follows I use it to find DMRs between monocytes and M0 macrophages samples in Blueprint. 
I considered the following samples:


monocytes: \texttt{C000S5A1bs},\texttt{C0010KA2bs},\texttt{C004SQ51},\texttt{C005PS51},\texttt{S000RD54},\texttt{S007G756}
M0 macrophages: \texttt{C005VG51},\texttt{S001S751},\texttt{S0039051},\texttt{S00BHQ51},\texttt{S00DVR51}


As an example, we consider on one side the 
first we consider all the positions where at least one of the samples has been measured
(WGBS loses positions dependeing on the vagaries of the high t sequencing: each sample
is usually measured ina slightly different set of positions).

Then we summarize the counts by first transforming them in an array of estimated methylation
values and then finding (by moment matching) a beta distribution which fits them.

The parameters of the Beta distribution can be interpreted as counts of 
a non converted and converted reads of the experiment giving the same methylation mean 
and variance as a maximum likelihood estimation from the available samples.

This trick allows me to convert a group of samples in a set of counts suitable to be gi
ven as input to \gimli{}.

To find the differentially methylated regions, I start by intersecting
( with \texttt{bedtools}\cite{bedtools} ) the segmentation computed for the first group
with the segmentation of the second : whenever two segments intersect I look
for the ratio between the difference in average methylation in the common area 
and the square root of the sum of the variances of the two samples; one can then
do a t-test.

To tue purpose of illustrating how it might work I plot here a DMR 
fromt the file mono.M0.l10.dmr  (coordinates :chr5	28927763	28928071).

Advantage : we don't need multiple samples per group, hence useful to control
replicates.




%\subsection{visualization}

%In figure \ref{ex1} I plotted the methylation in blocks (with $\lambda=1000$) 
%across a region of $1$ for $2$ Blueprint samples
%(G199,G202), together with the raw data.  Although some segments
%correspond to variable strectches, the reduction in complexity achieved by \gimli makes it 
%possible to observe some relevant signal by eye (in this case there is 
%an apparent loss of methylation across the $2$ samples, which are different 
%stages in the hematopoietic process).

%\begin{figure}\label{ex1}
%\includegraphics[width=10cm,angle=270]{G199.G200.G201.G202.chr1.gimli.eps}
%\caption{output of \gimli, for $\lambda=1000$ for two Blueprint samples. The loss of
%methylation is clearly visible.}
%\end{figure}

%\subsection{comparison across samples (DMRs)}

%\gimli  makes it easy to find regions which have different 
%methylations across samples. One needs to find segments 
%from the first and the second batch which intersect spatially
%but correspond to non overlapping intervals of methylation 
%(the maximum value of one 
%segment is less than the minimum of the other).

%Users can fine tune this procedure according to what exactly they 
%are looking for, taking into account that a standard deviation of $0.2$
%on the methylation measurements done with WGBS is not unusual, hence
%DMRs must be substantially separated to avoid confusion with sampling error.

%Also notice that I am not giving any algorithm to compute $p$-values
%or anything of the sort; first of all the credibility of each segment
%is measured by its log likelihood; secondly users can choose statistical
%tests suited to the specific signal (and the specific scale)
%they are considering.

%\begin{figure}\label{ex1}
%\includegraphics[width=10cm,angle=270]{G199.G202.100.200.dmr.eps}
%\caption{Differentially methylated regions : I considered segments 
%(found setting $\lambda=100$) longer than $200$bp 
%corresponding to
%non overlapping intervals of methylation values (ie the maximum value of one 
%segment is less than the minimum of the other).}
%\end{figure}

%\subsection{comparison with other kind of data}

%Many dataset in genomics are naturally expressed as segments. The
%fact that \gimli also generates annotated blocks allows one to
%integrate methylation data with other kind of information easily
%using such packages as, for instance, \texttt{bedtools} (XXX).

%As an example, I consider the set of $5227$ transcription start sites
%(TSSes) on human chromosome $1$. I infer the TSSes from a list of 
%regions used for gene expression quantification by the Blueprint 
%consortium. For each of them, I also consider the RPKM of the corresponding
%gene in the sample $C004GD$. I then compute with \gimli the segmentation
%of the methylation measurements for chromosome $1$ of $C004GD$.

%My intent is to show that the segments overlapping the TSSs
%say something about the transcriptional activity of the nearby gene.
%Accordingly,
%I use \texttt{bedtools} to intersect the list of blocks produced by \gimli
%with the list of transcription start sites; since more than one \gimli
%block can overlap a given TSS (blocks with different $\lambda$s can 
%cover the same region) I choose the largest segment covering the TSS
%with the constraint thet the maximum and minimum value of methylation
%in the block must not differ by more than $0.3$.

%I then compare the distribution of RPKMs for genes corresponding
%to TSS covered by segments with $\hat{\theta}<0.5$ with the same distribution
%for genes adjacent to TSS covered by blocks with $\hat{\theta} \geq 0.5$

%\begin{figure}\label{ex3}
%\includegraphics[width=7cm,angle=270]{boxplot_example_3.eps}
%\caption{Segments with different $\hat{\theta}$ overlapping the TSS correspond
%to different distribution of the RPKMs of the corresponding genes.}
%\end{figure}

%\subsection{Demarcation of unusual regions} 

%Next I used \gimli  to look for regions where the 
%methylation
%ratio goes from high to low and to high again.

%In chromosome $1$ of G199, $972$ such intervals intersect one (or more) 
%TSSes. I haven't annotated the remaining ones, but in my view they constitute
%an interesting pattern to look at.

\section{Implementation}

\gimli{} is written in \texttt{C}, and the source
dode is available from the github page of the author. 
On a Pentium(R) Dual-Core  CPU (\texttt{E5400@2.70GHz})
with $2048$ KB of cache it takes $\approx 840s$ to process the 
methylation levels of the sample \texttt{C001UYA3bs} ($23815993$ CpGs) 
with $4$ different $\lambda$s ($1,10,100$ and $1000$).
This includes the time for decompressing the input file and selecting some of its columns
with \texttt{awk}.


\bibliography{gimli_paper}
\bibliographystyle{plain}
\end{document}
