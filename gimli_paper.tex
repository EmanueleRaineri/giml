\documentclass[11pt]{amsart}
\usepackage{graphicx}
\newcommand{\lik}{\ensuremath{\mathcal{L}}}
\newcommand{\gimli}{\texttt{gimli}}

\begin{document}
\title{Multiscale segmentation of methylation data}
\author{Emanuele Raineri}
\date{\today}
\maketitle

\begin{abstract}
Due to the growing numbers of tracks of numerical values attached to the 
human genome, especially in the field of epigenetics measurements 
(chromatin modification,
Hi-C, whole genome bisulfite sequencing, and more), 
it is useful to implement software 
which can quickly summarize the data and extract patterns to help 
automating at least  part of the biological analysis. 
In this paper, I show one efficient way of computing 
segmentations of methylation values determined by 
whole genome bisulfite sequencing
(hence inferred from the number of converted and non converted reads 
at each position).  
A segmentation joins adjacent position
which have similar properties, while the boundary between segments 
indicates more or less abrupt changes
in the signal which might relate to biological mechanisms. 
For example
a drop in methylation in a promoter area is is some cases associated with 
transcriptional activity of the corresponding gene; 
segments with intermediate values might be related to imprinted regions; 
and so on and so forth.
One salient aspects which must be considered when it comes to 
analyzing methylation dataset is that this epigenetic mark seems 
to have effect at various genomic scales 
(ranging from hundred of bases to megabases), hence a multi scale method 
is called for. 
Here I introduce a fast probabilistic model which, coupled with a known greedy 
algorithm for copy number detection 
(of which I rewrote the implementation from scratch) 
calculates automatic segmentations at different scales.
I also present some concrete examples of its utility. 
In particular I will show how this 
kind of statistical modeling is very useful for 
visualization, comparison across samples,
comparison with other kind of data which are naturally expressed as a set of 
segments (e.g. chromatin modifications),  and demarcation of unusual regions in 
order to direct further analysis (i.e. regions with stark variations in the signal).
\end{abstract}

\section{introduction}

Whole genome Bisulfite sequencing (WGBS) allows measurements of methylation 
status with single base resolution across an entire genome. This tecnique shows 
that the genome contains some loci corresponding to sudden changes in methylation 
and many more regions where this epigenetic mark has a strong local
correlation, in the sense the the methylation status at one position predicts 
very well the levels nearby. 
This local correlation has been observed many times ( for example
in the Bsmooth paper \cite{bsmooth}) and is confirmed in 
figure~\ref{corr} which I computed 
first selecting at random $10^6$ pairs of  methylated loci; then binning
them according to the distance between the members of the pair (I considered 20 
bins corresponding to distance between $0$ and $100$bp,$100$ and $200$bp, 
etc\dots);
finally computing for each bin the correlation in methylation values 
across the pairs contained in it.

\begin{center}
\begin{figure}\label{corr}
\includegraphics[width=4cm,angle=270]{out.correla.eps}
\caption{This plot is produced by extracting $1$ million random pairs of CpG loci;
binning the pairs in $20$ different bins  according to the distance between the members
of the pair (from $0-100$ to $1900-2000$); and computing the correlation in each bin,
using the pairs contained in it.}
\end{figure}
\end{center}

Another interesting aspect of methylation is that it possesses some sort of scale 
invariance: variations can be observed both at the level of a single promoter
region (hundred of bases, see for example XXX) and when looking at stretches
of millions of bases (cite XXX).
This entails some problems when analyzing this kind of data: a rough
running average with a window of many megabases might destroy interesting
local detail, but at the same time large scale (slow) changes are also
significant and must be looked at.
I present  a software (\gimli) which computes a multiscale segmentation 
of DNA methylation 
data acquired through WGBS and helps making sense of the data and compare 
it across samples.

A segmentation is a statistical model of a dataset.
In general, when building a statistical model one tries to find a reasonable 
trade off between two 
features of it: its goodness of fit and its complexity.
In our case these two aspects have a very simple interpretation : the goodness 
of fit increases with the number of segments $N_S$ used to describe the 
methylation data whereas the complexity of the model decreases with $N_S$.  
At one extreme, representing the complete dataset with one big segment would 
give a very simple model with the worst possible fit and 
at the other using many segments of length $1$ would give us a perfectly 
fitting but very complex summary of the data.
The complexity of the model is linked to the typical scale of the segments used
in the sense that if the typical block is very long one needs few of them to 
cover the genomic region under consideration.

\section{the algorithm}

\gimli's algorithm consists of two parts : 
a formula to evaluate quickly the 
goodness of fit of any given segmentation
and a protocol to decide (greedily) which  segments to join depending on
\begin{itemize}
\item{}a coarseness parameter $\lambda$ and on 
\item{}the distance measured in bp along
the genome between the segments been considered for merging. 
\end{itemize}
The second part uses and  algorithm previously described   in 
the context of copy number variation detection (\cite{vega}, which is in 
turn an adaptation 
of a 2 dimensional image analysis algorithm due to Mumford and Shah, 
\cite{mumfordshah}). Finally, the segmentation itself, together with some 
helpful statistics, is printed out for the use to analyze.

The algorithm takes in input a set of $N$ methylated positions 
(CpGs or not CpGs) indexed by $j=1,\dots,N$ 
(for human chromosome $1$ and at a 
decent coverage one has $N \approx 2E6$).
and returns a set of blocks covering the same loci.
By block (or segment) here I mean either a single position or a collection
of contiguous positions. The way it works is by sweeping
iteratively the current list of segments and evaluate whether any pair 
of adjacent blocks can be conveniently merged into one. So how does one gone 
about deciding whether to merge? This is  done by checking whether the loss in 
likelihood consequent to the merging is compensated (or more than compensated)
by the reduction in complexity achieved by it.

One also has a collection of $N_S$ 
segments (indexed by $i=1,\dots,N_S$), 
which at the beginning
coincide with the methylated positions ( i.e. it is a collection of $N_S$ 
segments of length $1$ with $N_S=N$.).
Finally, one needs to assign a value to a parameter, $\lambda$ which controls the 
complexity of the final segmentation.

The algorithm works by trying iteratively to join segments until this
is no longer possible; the decision is taken looking at the change
in the total likelihood generated by the merging of two given segments. 

I measure the likelihood of a segmentation as the sum of the log likelihoods
of all its segments. In turn the log likelihood of a block is defined by thw
two following formul\ae:

If the
segment coincides with a single locus, 
the log likelihood  is evaluated as 

\[\lik_i=\log {\overline{n}_i+n \choose \overline{n}_i} +
	{\overline{n}_i}\hat{\theta}_i+
	n_i(1-\hat{\theta}_i)
\label{loglik}

where

$n_i$ and $\overline{n}_i$ are the converted and unconverted reads respectively 
and

\[\hat{\theta}_i=\frac{\overline{n}_i}{\overline{n}_i+n_i}\]

is the maximum likelihood estimation for the parameter of the binomial process 
which generates $\overline{n}_i$.

If (more in general) a block contains  more than a single CpG  
(think for concreteness of a segment which includes all the positions from 
$j_1$ to $j_2$) is 
computed as in equation (\ref{loglik}) except for $\hat{\theta}$ which is now:

\[
\hat{\theta}=\frac{\overline{n}}{n + \overline{n}}
\]

where

\[
	\overline{n}=\sum_{j=j_1}^{j_2} \overline{n}_j}
\]

and

\[
	n=\sum_{j=j_1}^{j_2} n_j}
\]


As said, the total likelihood of the segmentation is given by 
$\mathcal{L}=\sum_i\mathcal{L}_i$.

Armed with this definition of likelihood (which is a proxy for goodness of fit) 
we can explain more precisely the following step of the algorithm:
for each pair of adjacent segments $(i,i+1)$ we can compute the loss in 
total likelihood that we get if we merge them
in one single segment. This is:

\[\Delta \lik=\lik-\mathcal{L}_i-\mathcal{L}_{i+1}+\mathcal{L}_{i,i+1}\]

notice that $\mathcal{L} \leq \mathcal{L}_1+\mathcal{L}_2$ hence this always 
results in a loss in total 
likelihood. This loss is compensated, though, by the fact the the number 
of segments decreases by $1$. To take into account this decrease in 
complexity we consider an adjusted likelihood change:
$\Delta \lik = \Delta \lik+\lambda$.

We look for the maximum $\tilde{\Delta \lik}$ and if it is positive, we merge the
corresponding segments into one segment.

We then repeat this operation until we can't merge any more pair of adjacent 
segments; in this case we can decide to increment 
$\lambda$ and try again, or to give up and exit the algorithm.

\subsection{sparsity of CpGs}

Methylated sites are irregularly spaced along the genome and the variability 
introduced by the sequencing process might further increase the distance 
between adjacent sites.
To avoid building segments which span long regions where no CpGs exist, one 
can multiply $\lambda$ by a penalty term of the
form $\exp(-\frac{D_{i,i+1}}{D_0})$ where $D_{i,i+1}$ is the distance 
between the the rigth end of segment $i$ and the left end of segment $i+1$ and 
$D_0$ is a constant which (looking at figure \ref{corr}) can be reasonably 
set to $1000$. 

Hence the final expression for the adjusted likelihood change is :

$\Delta \lik = \Delta \lik+\lambda \exp(-\frac{D_{i,i+1}}{D_0}) $.

I can now refine quantitatively the assertion that the complexity of the model
is linked to the typical lengths of the segments : as figure \ref{boxplot1} 
shows the median length of the blocks increases with $\lambda$. This is 
accompanied by a decrease in likelihood of the segments (\ref{boxplot2})

\begin{figure}\label{boxplot1}
\includegraphics[width=4cm,angle=270]{boxplot1.eps}
\caption{Median lenght of blocks is higher at higher $\lambda$s.}
\end{figure}

\begin{figure}\label{boxplot2}
\includegraphics[width=4cm,angle=270]{boxplot2.eps}
\caption{Likelihood of blocks diminishes with $\lambda$.}
\end{figure}


\section{examples}

While one can rigorously test that the numbers computed by \gimli are in 
accordance with the description
of the algorithm given in this paper, the purpose of gimli is to help 
detect automatically interesting
biological patterns in methylation data, and this can't be tested in any 
precise sense. To show that \gimli is a useful tool
though, I will give 4 examples of analyses which can be produced very quickly
using it. 

\subsection{visualization}

In figure \ref{ex1} I plotted the methylation in blocks (with $\lambda=1000$) 
across the whole of chromosome $1$ for $4$ Blueprint samples
(G199,G200,G201,G202). Although the plot represents around $2$ millions of 
CpGs per each sample, the reduction in complexity achieved by \gimli makes it 
possible to observe by eye a lot of interesting signal (in this case there is 
an apparent loss of methylation across the $4$ samples, which are different 
stages in the hematopoietic process).

\begin{figure}\label{ex1}
\includegraphics[width=4cm,angle=270]{G199.G200.G201.G202.chr1.gimli.eps}
\caption{output of \gimli, for $\lambda=1000$ for four Blueprint samples. The loss of
methylation is clearly visible.}
\end{figure}

\subsection{comparison across samples (DMRs)}

It is easy to find regions which have starkly different 
methylations across samples. One needs to find segments 
from the first and the second batch which intersect spatially
but correspond to non overlapping intervals of methylation 
(the maximum value of one 
segment is less than the minimum of the other).

Users can fine tune this procedure according to what exactly they 
are looking for, taking into account that a standard deviation of $0.2$
on the methylation measurements done with WGBS is not unusual, hence
DMRs must be substantially separated to avoid confusion with sampling error.

Also notice that I am not giving any algorithm to compute $p$-values
or anything of the sort; first of all the credibility of each segment
is measured by its log likelihood, secondly users can choose statistical
tests suited to the particula signal (and the particular scale)
they are considering.

\begin{figure}\label{ex1}
\includegraphics[width=4cm,angle=270]{G199.G202.20.200.dmr.eps}
\caption{Differentially methylated regions : I considered segments corresponding to
non overlapping intervals of methylation values (ie the max value of one 
segment is less than the minimum of the other).}
\end{figure}

\subsection{comparison with other kind of data}

There are many experiments in genomics the results of which are naturally 
expressed as a set of segments (e.g. detection of chromatin modifications)

I considered the chromatin segmentation on chromosome $1$ of the
Blueprintsample C004GD. Extracted only the segments which correspond
to the state of active promoter (there are $8861$ such segments, over a total
of $28479$).

Now, I expect that the methylation corresponding to active promoters should be
low : this appears not to be the case. Infact, when one intersects the active
promoters as defined by the Blueprint segmentation and the blocks of CpGs as 
defined by \gimli (I also ask that the intersection between two segments covers
at least $50\%$ of the chromatin state, and that he difference between
the maximum and minimum methylation in a block is $\leq 0.2$) one finds
that the distribution of the maximum likelihood $\theta$s for the 
methlyated blocks has a median of $0.82$. This is puzzling and some 
to be inverstigated more thoroughly in the near future and shows how \gimli
can be used for checking quickly the consistency of a composite dataset.

\subsection{Demarcation of unusual regions} 

Most of the CpGs in the human genome have either methylation $1$ or
methylation $0$. This implies that those CpGs are methylated in the same
way in all the cells of the sample used to run the WGBS protocol
(of course in a single cell every position has either methylation $0$ or $1$:
intermediate values in the measurements are given by the superposition of 
many cells).

Now, to show how \gimli can help in the search for unusual regions which
can inspire further investigation, I chose to look at sequences of blocks 
which correspond to dips in the signal, regions where the methylation
ratio goes from high to low and to high again.

\section{implementation}

\gimli is written in \texttt{C}, and the source
dode is available from the github page of the author. Together
with \gimli I also distribute \texttt{libgimli.R},
a set of R functions which can be used to produce plots and statistical
analyses based on \gimli's output.
\bibliography{gimli_paper}
\bibliographystyle{plain}
\end{document}
